\chapter{Design}
The overall design to achieve the goal on section XX is presented in this chapter. The prerequisites and criteria are firstly examined to replicate the rehabilitation scenario.

\setlength{\tabcolsep}{18pt}
\renewcommand{\arraystretch}{1.5}
\begin{table}[ht]
\caption{Comparison of Two Simulation Studies}
\label{tab:Comparison}
\resizebox{\textwidth}{!}{%
\begin{tabular}{|l|l|}
\hline
\multicolumn{1}{|c|}{\textbf{\begin{tabular}[c]{@{}c@{}}Developing a Quasi-Static Controller\\  for Paralyzed Human Arm\end{tabular}}} &
  \multicolumn{1}{c|}{\textbf{\begin{tabular}[c]{@{}c@{}}Developing an EMG-Based Quasi-Static\\  Controller for Stroke Reaching Rehabilitation\end{tabular}}} \\ \hline
Focused for spinal cord injury & Focused on stroke survivors                                                                             \\ \hline
Controlling 9 muscles          & Controlling 2 muscles                                                                                   \\ \hline
Surgical FES at 13Hz           & External FES stimulation                                                                                \\ \hline
Exoskeleton Arm Support        & \begin{tabular}[c]{@{}l@{}}No Exoskeleton Arm\\  (enough mobility provided by the patient)\end{tabular} \\ \hline
Wrist Position Control         & EMG and Wrist Position Control                                                                          \\ \hline
Multiple movements in the workspace &
  \begin{tabular}[c]{@{}l@{}}Same target, different initial positions. \\ Repetition of movement for rehabilitation purposes.\end{tabular} \\ \hline
\end{tabular}%
}
\end{table}
\section{Design Criteria}

To pinpoint the crucial factors that would direct our choice of design and tools, a thorough criterion analysis was carried out. The following important standards came to be recognized as essential elements in the search for an efficient simulation solution:

\begin{enumerate}
    \item \textbf{Real Arm Simulator} In order to correctly reproduce real-world dynamics and movements, an exact simulation of an arm was required.
    \item \textbf{Accurate Dynamics Representation} To ensure that the simulated arm's behavior and responses adhere to the fundamentals of human biomechanics.
    \item \textbf{Neural Stimulation Control} An essential criteria to provide more realistic interaction between the simulated arm and the brain or the FES.
    \item \textbf{Tracking} Ability to track the arm´s movement either in joint space or task space. This will provide flexibility in the control system.
    \item \textbf{EMG Reading} It should present the capability or a possibility to read real or fake neural excitation data. This data will be used as the main driven input to control the arm's stimulation.
    \item \textbf{Control Model Integration} A key criterion is that the tool presents the possibility to develop and evaluate various control methods.
    \item \textbf{External Forces Addition} In order to simulate realistic interactions and enabling more thorough and accurate simulation of arm dynamics.
    
\end{enumerate}



\section{Block Diagram}
\section{Simulation Experiment Setup}
% Muscle Selection and Modelling
Spasticity (velocity-dependent stiffness) in stroke typically produces resistance to arm extension due to overactivity of biceps, wrists and finger flexors and loss of activity of triceps, anterior deltoid, wrist and finger extensors (reference 24). \cite{IOL}
Triceps and anterior deltoid are hence selected for stimulation to align with the clinical need to increase muscle tone and restore motor control of weakened muscles. 
The relationship between muscle stimulation adn subsequent movement are well explored. However, simplification opens up routes for both aprameter identification and controller derivation that have not yet been possible for more complex models. \cite{IOL}
It is assumed that applying FES to the anterior deltoid produces a moment about an axis that is fixed with respect to the shoulder. Applying FES to the triceps produces a moment about an axis orthogonal to both the forearm and upper arm.\cite{IOL}

As discussed in [26] the most prevalent form of muscle representation is the Hill-type muscle. (Explain the Hill-Type muscle)
% FES deviced used
Functional Electrical Stimulator (a DSP-based FES) was developed in our laboratory. 23,35,50Hz are available ofr the pulse frequency. 
The pulse width ranges from 0 to 300 $\mu$s. The output current is from 0 to 100 mA. The stimulation wave can be modified into two models: single phase, and biphasic phase. \cite{NNPID}




