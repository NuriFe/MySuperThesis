\chapter{Design}
The overall design to achieve the goal on section XX is presented in this chapter. The prerequisites and criteria are firstly examined to replicate the rehabilitation scenario.

\section{Block Diagram}
\section{Criteria Analysis}

To pinpoint the crucial factors that would direct our choice of design and tools, a thorough criterion analysis was carried out. The following important standards came to be recognized as essential elements in the search for an efficient simulation solution:

\begin{enumerate}
    \item \textbf{Real Arm Simulator} In order to correctly reproduce real-world dynamics and movements, an exact simulation of an arm was required.
    \item \textbf{Accurate Dynamics Representation} To ensure that the simulated arm's behavior and responses adhere to the fundamentals of human biomechanics.
    \item \textbf{Neural Stimulation Control} An essential criteria to provide more realistic interaction between the simulated arm and the brain or the FES.
    \item \textbf{Tracking} Ability to track the arm´s movement either in joint space or task space. This will provide flexibility in the control system.
    \item \textbf{EMG Reading} It should present the capability or a possibility to read real or fake neural excitation data. This data will be used as the main driven input to control the arm's stimulation.
    \item \textbf{Control Model Integration} A key criterion is that the tool presents the possibility to develop and evaluate various control methods.
    \item \textbf{External Forces Addition} In order to simulate realistic interactions and enabling more thorough and accurate simulation of arm dynamics.
    
\end{enumerate}



