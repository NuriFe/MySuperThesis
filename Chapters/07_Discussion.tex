\chapter{Discussion and Future Work}

Stroke is a leading cause of disability, particularly among the elderly. There will be a greater demand for stroke rehabilitation because estimates point to an increase in the older population. In particular rehabilitation of the upper limbs, as arm hemiparesis is a common aftermath of stroke. The economic implications of stroke are profound, largely stemming from the high costs associated with post-stroke care and rehabilitation therapies. This emphasizes the need for cost-effective rehabilitation strategies. This project focuses on FES rehabilitation, where electrical stimulation is used to improve muscle recovery. Task-specific exercises with training programs, especially when encouraging patient engagement, lead to better recovering results. Consequently, this project is centered on "reaching task" exercises, where EMG sensors regulate the FES device, adjusting the electrical current based on muscle activity, thereby promoting patient involvement. The goal of this thesis is to create a simulation framework tailored for testing a range of EMG-based control methods, particularly focusing on the repercussions of stroke on an arm's reaching movement.

The research journey can be delineated into three primary stages of controller design, each iteratively building upon its predecessor. Each controller was tested over 29 reaching positions spaced every 0.01m. The foundational stage was the Static Controller, aimed at ensuring arm stability at a specific point. In this phase, two critical Semi-Parametric Gaussian Process Regression models were developed: the Inverse Statics model and the Muscle Torque Production model. The former calculates the essential joint torque for a given static arm position, while the latter associates arm configuration and muscle activation with the resulting joint torque. Torque values were derived by converting the force required to maintain static positions, both with and without muscle activation, through the Kinematic Jacobian. This force determination involved a PI-tuned controller. By merging the Inverse Static, which provides the desired static torque, with a refined PI controller supplying the corrective torque, the Static Controller generates the required torque to sustain the arm. This torque is converted into muscle activation by optimizing the equation that relates them, using the mapping matrix from the Muscle Torque Production model. The Static Controller's performance was measured across 960 distinct points with a success rate of 74.06\%. When tested over the 29 reaching positions the Static Controller maintained the position with minimal variance. It showcased high accuracy in the x and z coordinates, but slightly less precision in the y coordinate.

The following phase introduced movement with the Path Following Quasi-Static Controller. Building on the feasible points from the prior stage, a navigational path was generated using the K-Nearest-Neighbours algorithm. This defined trajectory directed the arm through a sequence of predetermined points. Similar to the Static Controller, there was greater accuracy and precision in the x and z coordinates compared to the y coordinate. However, the target positions across the 29 designated spots were achieved with notable success.

The last part of this research was marked by the introduction of the EMG-Influenced FES controller. In this stage, muscle activations were strategically modulated to mimic a stroke-like movement pattern, primarily by overactivating the biceps. This phase introduced a simulated FES device, with its control parameters dynamically adjusted based on real-time neural activity data, specifically the triceps. The purpose of the FES device is to adjust the muscle excitation directed at the triceps. This controller was assessed across the predetermined reaching points at three distinct stroke severity levels. It maintained consistent accuracy and precision across these stroke levels, mirroring the pattern observed in previous controllers where performance in the y-direction was slightly less optimal. Nevertheless, this controller effectively introduced the required neural stimulation to the arm, aiding it in following the intended trajectory. Notably, oscillations were observed when the FES controller was incorporated, likely because the employed controller lacked both derivative and integrator components. Still, this achievement solidly establishes a foundation for the development and deployment of superior EMG-based controllers that could be integrated into real-world FES scenarios.

This research journey culminated in the successful deployment of an EMG-influenced FES controller, setting the stage for further exploration of EMG-based FES controllers. The potential implications are profound, suggesting avenues for creating more efficient, accessible, and patient-centric solutions.

Despite its successes, this study acknowledges its constraints. Primarily, the study was limited to the control of a single muscle, with the EMG-Based FES controller's design being influenced solely by one muscle's excitation. Additionally, challenges may arise when translating these findings to a real-world FES implementation. To achieve a more accurate depiction an authentic recruitment model, translating the required muscle activations into the FES device's current generation, is essential. There is also a clear need for refining the tuning of stroke parameters, potentially achieved through comparative analyses with actual EMG data.

Looking ahead, several promising avenues may be explored.  This work has laid the groundwork toward experimenting with diverse EMG-based FES controllers for upper arm rehabilitation. Some possible controllers will include PID, bio-inspired controllers, to artificial neural network control approaches. Expanding the study to encapsulate multiple muscles and a broader array of movements is also a clear next step, as is the integration of real-world EMG data. Central to this study is its pioneering strategy of controlling the DAS via neural excitations, introducing a realm of untapped potential. While the MATLAB-centric approach enhances accessibility, it isn't without its limitations, such as integration with Simulink. Nonetheless, this research stands as a bedrock for future advancements, providing a comprehensive literature for understanding of the DAS with the objective of leveraging this knowledge for innovative, patient-centric solutions.

\chapter{Conclusion}

The goal of this thesis was to devise a simulation framework for the exploration of EMG-based FES control methods, emphasizing the impact of stroke on an arm's reaching movement. This was motivated by the growing need for efficient stroke rehabilitation for the increasing elderly population. Through an intricate three-stage process, the research evolved from stabilizing arm positions using the Static Controller to inducing movement with the Path Following Quasi-Static Controller, culminating in the development of the EMG-Influenced FES controller. This latter achievement, particularly notable for its potential in simulating stroke-like movement patterns, encapsulates the essence of the thesis. Although the research acknowledges limitations like focusing solely on a single muscle's excitation and single muscle's activity readings, its findings spotlight the profound potential of EMG-controlled FES for future patient-centric rehabilitation solutions. The exploration in this study sets a solid foundation for further advancements in this domain, promising more inclusive, efficient, and effective therapeutic interventions.