\section*{Abstract}
\addcontentsline{toc}{section}{Abstract}
Stroke is a leading cause of disability, especially among the elderly, escalating the demand for effective stroke rehabilitation due to anticipated growth in the ageing population. This research concentrates on \ac{FES} rehabilitation for muscle recovery, emphasizing the "reaching task" exercises. Within this framework, \ac{EMG} sensors dynamically modulate the \ac{FES} device, fostering patient engagement. The research delineated a three-stage approach to EMG-Based FES controller design. Initially, the Static Controller was devised, anchored on two crucial Semi-Parametric \ac{GPR} models, emphasizing arm stability. Subsequently, the Path Following Quasi-Static Controller was introduced, leveraging the \ac{KNN} algorithm to facilitate movement through predetermined trajectories. The last stage of the research manifested in the \ac{EMG}-Influenced \ac{FES} controller, modulating muscle activations to simulate stroke-induced patterns, accentuated by the real-time control of a simulated \ac{FES} device. This controller showcased consistency across varying stroke severities, marking a significant step towards real-world \ac{FES} \ac{EMG}-based control applicability. The study faced inherent constraints such as the singular muscle-focused control, the use of only one muscle to influence the controller, and potential challenges in real-world translation. This work has explored diverse avenues for upper arm rehabilitation through simulation of \ac{EMG}-based \ac{FES} controllers, with innovative control of \ac{DAS} via neural excitations. The project was developed in MATLAB facilitating ease of access. It will enable researchers to efficiently design and simulate their EMG-driven controls for \ac{FES} devices. This thesis ultimately seeks to revolutionize rehabilitation methods, steering them towards a more patient-centric approach.





