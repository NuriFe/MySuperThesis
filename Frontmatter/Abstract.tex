\section*{Abstract}
\addcontentsline{toc}{section}{Abstract}
Stroke is a leading cause of disability, especially among the elderly, escalating the demand for effective stroke rehabilitation due to anticipated growth in the aging population. This research concentrates on Functional Electrical Stimulation (FES) rehabilitation, employing electrical stimulation for muscle recovery, emphasizing the "reaching task" exercises. Within this framework, EMG sensors dynamically modulate the FES device, fostering patient engagement. The research delineated a three-stages approach to controller design. Initially, the Static Controller was devised, anchored on two crucial Semi-Parametric Gaussian Process Regression models, emphasizing arm stability. Subsequently, the Path Following Quasi-Static Controller was introduced, leveraging the K-Nearest-Neighbours algorithm to facilitate movement through predetermined trajectories. The last stage of the research manifested in the EMG-Influenced FES controller, modulating muscle activations to simulate stroke-induced patterns, accentuated by the real-time control of a simulated FES device. This controller showcased consistency across varying stroke severities, marking a significant step towards real-world FES EMG-based control applicability. Nevertheless, the study faced inherent constraints, such as the singular muscle-focused control and potential real-world translation challenges. This work has pioneered avenues for diverse EMG-based FES controller simulation exploration for enhanced upper arm rehabilitation. Central to this is the innovative control of the Dynamic Arm Support (DAS) via neural excitations. While rooted in MATLAB, paving the way for greater accessibility, there remains room for further expansion and refinement in future research endeavors.





